% !TEX encode = UTF-8
% !TEX program = xelatex
% \documentclass[answer]{THUExam}


% =================================================
%       用来生成A3卷子的代码
% =================================================
% \documentclass[a3paper]{article}
% \usepackage{geometry}
% \usepackage{pdfpages}
% \geometry{
% 	layout = a3paper,
% 	landscape,
% 	a3paper
% }
% \begin{document}
% \includepdf[pages=-,nup=2x1]{THUExam-demo}
% \end{document}



\PassOptionsToPackage{quiet}{xeCJK}
\documentclass[answer]{THUExam}
% \documentclass[]{THUExam}
\usepackage{HTNotes-math}
\usepackage{pifont}

\usepackage{tikz}
\usetikzlibrary{positioning, calc, shapes.geometric, shapes, shapes.multipart, arrows.meta, arrows, decorations.markings, external, trees}

% =================================================
%       PDF信息
% =================================================
% PDF信息里的标题栏
\title{清华大学 2021–2022 学年 秋季学期 离散数学 1 }
% PDF信息里的作者栏
\author{杨维铠}
% PDF信息里的主题
\Subject{Exam}
% PDF信息里的关键词
\Keywords{离散数学}

% =================================================
%       试卷头信息
% =================================================
% 试卷头里的年份
\Year{2021--2022}
% 试卷头里的学期
\Semester{秋季}
% 试卷头里的课程
\Course{离散数学1}
% 考试时间
\Time{2022年1月8日14:30-16:30}
% 试卷头里的类型,如A/B/模拟等
\Type{A}

\begin{document}
\vspace{1cm}
% 生成试卷表头
\makehead

\vspace{1cm}

\makepart{单选题}{每小题~2~分,共~4~分}在下列各小题中选择其中\fbox{一个答案},标注在题目中的括号内. 


\begin{problem}
本题答案为A\pickout{A}
\options
{选项A}
{选项B}
{选项C}
{选项D}
\end{problem}

\begin{problem}
本题答案为C\pickout{C}
\options
{一个比较短的选项A}
{一个比较短的选项B}
{一个比较长长长长长长长长长长的选项C}
{一个比较短的选项D}
\end{problem}

\begin{problem}
本题答案为D\pickout{D}
\options
{一个非常长长长长长长长长长长长长长长的选项A}
{一个非常长长长长长长长长长长长长长长的选项B}
{一个非常长长长长长长长长长长长长长长的选项C}
{一个非常长长长长长长长长长长长长长长的选项D}
\end{problem}

\vspace{2cm}

\makepart{判断题}{每小题~1~分,共~2~分}在题目中的括号内标出\ding{51}或\ding{55}.

\begin{problem}
该命题为真命题.\pickout{\ding{55}}
\end{problem}

\begin{problem}
该命题为假命题.\pickout{\ding{51}}
\end{problem}

\makepart{填空题}{每小题~2~分,共~2~分}.

\begin{problem}
正多面体一共有\fillin{$\quad 5\quad $}种.
\end{problem}

\newpage
\makepart{解答题}{共~4~分}题目说明为\fbox{画出}或\fbox{写出}的,只用写答案,不用写过程.

\begin{problem}
(4分)证明一个代数结构是群需要证明那些性质
\end{problem}
\bigskip
\begin{solution}
\begin{itemize}
    \item 封闭性\dotfill 1分
    \item 结合律\dotfill 2分
    \item 有幺元\dotfill 3分
    \item 有逆元\dotfill 4分
\end{itemize}
\end{solution}
\begin{note}
\begin{itemize}
    \item 每漏一条扣一分. 
\end{itemize}
\end{note}
\end{document}
